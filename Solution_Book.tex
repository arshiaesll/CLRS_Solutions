\documentclass{book}
\usepackage{graphicx} % Required for inserting images

\title{CLRS Solutions}
\author{aeslami }
\date{September 2024}

\begin{document}

\maketitle

\chapter{The Role of Algorithms in Computing}

\section{Section 1.1}

\subsection{Exercise 1.1-1}
\begin{enumerate}
    \item An example for sorting is sorting a list of Names in alphabetical order.
    \item An Example for comptuting a convex hull is finding the diameter of a set of points

\end{enumerate}

\subsection{Exercise 1.1-2}
\begin{enumerate}
    \item Memory is another important resource. Another example is thermal energy 
    produced by computers.
\end{enumerate}

\subsection{Exercise 1.1-3}
\begin{enumerate}
    \item Linked list is a example of a data structure. It's advantage is that an 
    item can be added or removed in O(1) time. A disadvantage is that it 
    doens't support random access.

\end{enumerate}
\subsection{Exercise 1.1-4}
\begin{itemize}
    \item The shortest path problem is mainly concerned about reaching a point 
    from another point with the minimum cost, but the traveling salesman problem 
    is about finding the shortest path that visits all of the points and returns 
    to the origin point with the minimum cost.
\end{itemize}

\subsection{Exercise 1.1-5}
\begin{enumerate}
    \item Finding the root of a polynomila is a example of a probelm that only 
    best solution will do.
    \item Finding a move in a game of Chess is a example of a problem that an 
    "approximate solution" is good enough.
\end{enumerate}
\subsection{Exercise 1.2-1}
In a navigation application, an algorithm is used to find the shortest path 
between two points.

$8 \times n^2 \leq{64 \times \lg{n}}$

$n^2 \leq{8lg{n}}$

$ 2 \leq{n} \leq{43}$

\subsection{Exercise 1.2-2}

$ n \leq{15}$

\subsection{Problem 1-1}


\begin{table}[h]
    \centering
    \begin{tabular}{|l|r|r|r|r|r|r|}
        \hline
        & 1 minute & 1 hour & 1 day & 1 month & 1 year & 1 century \\
        \hline
        $\lg n$     & $2^{60000000}$ & $2^{3600000000}$ & $2^{86400000000}$ & $2^{2592000000000}$ & $2^{31536000000000}$ & $2^{3153600000000000}$ \\
        $\sqrt{n}$  & $3.6 \times 10^{12}$ & $1.296 \times 10^{16}$ & $7.46 \times 10^{18}$ & $6.72 \times 10^{21}$ & $9.95 \times 10^{23}$ & $9.95 \times 10^{27}$ \\
        $n$         & $6 \times 10^7$ & $3.6 \times 10^9$ & $8.64 \times 10^{10}$ & $2.592 \times 10^{12}$ & $3.1536 \times 10^{13}$ & $3.1536 \times 10^{15}$ \\
        $n \lg n$   & $2.8 \times 10^6$ & $1.3 \times 10^8$ & $2.0 \times 10^9$ & $4.9 \times 10^{10}$ & $5.4 \times 10^{11}$ & $3.9 \times 10^{13}$ \\
        $n^2$       & $7745966$ & $60000000$ & $293938769$ & $1609968129$ & $5615692821$ & $56156922861$ \\
        $n^3$       & $391420$ & $1532278$ & $4420825$ & $13736056$ & $31593173$ & $146645033$ \\
        $2^n$       & $25$ & $31$ & $36$ & $41$ & $44$ & $51$ \\
        $n!$        & $12$ & $13$ & $14$ & $15$ & $16$ & $17$ \\
        \hline
    \end{tabular}
    \caption{Maximum size of n for different time complexities and time limits (assuming 1 operation per microsecond)}
    \label{tab:time_complexity}
\end{table}

\chapter{Getting Started}

\section{Insertion Sort}

The following is a Python implementation of the insertion sort algorithm. This code demonstrates the basic structure and functionality of insertion sort:

\begin{verbatim}
def insertion_sort(arr):
    """Perform insertion sort on the given array.
    Input: arr - A list of comparable elements
    Output: The same list, sorted in ascending order
    Time complexity: O(n^2), Space complexity: O(1)"""
    for j in range(1, len(arr)):
        key, i = arr[j], j - 1
        while i >= 0 and arr[i] > key:
            arr[i+1] = arr[i]
            i -= 1
        arr[i+1] = key
    return arr

def main():
    test_array = [5, 2, 4, 6, 1, 3]
    print("Original array:", test_array)
    sorted_array = insertion_sort(test_array)
    print("Sorted array:", sorted_array)

if __name__ == "__main__":
    main()
\end{verbatim}



\end{document}

